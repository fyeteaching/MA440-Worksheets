\usepackage[most]{tcolorbox}
\tcbuselibrary{theorems, hooks}

\newcommand{\proofname}{Proof}
\newcommand{\definitionname}{Definition}
\newcommand{\theoremname}{Theorem}
\newcommand{\lemmaname}{Lemma}
\newcommand{\propositionname}{Proposition}
\newcommand{\corollaryname}{Corollary}
\newcommand{\examplename}{Example}
\newcommand{\exercisename}{Exercise}
\newcommand{\remarkname}{Remark}
\newcommand{\solutionname}{Solution}
\newcommand{\howtoname}{How-to}
\newcommand{\notename}{Note}

\newcommand{\thmcnt}{section}

\newlength{\normalparindent}
\setlength{\normalparindent}{\parindent}

\tcbset{
  common/.style={
    enhanced,
    breakable,
    % frame hidden,
    % opacityframe=.6,
    colback=white,
    coltitle=blue!90,
    grow to left by=1em,
    grow to right by=1em,
    left*=0pt,
    right*=0pt,
    boxrule=1pt,
    titlerule=0mm,
    arc=5pt,
    fonttitle=\upshape\bfseries,
    theorem style=plain,
    before upper app={\setlength{\parindent}{\normalparindent}},
    },
  defstyle/.style={
    % colback=green!10!white,
    colframe=green!80!black,
  },
  thmstyle/.style={
    fontupper=\itshape,
    % colback=red!10!white,
    colframe=red!75!black
  },
  exmstyle/.style={
    frame hidden,
    colback=blue!5!white,
    after app={\vspace{\stretch{1}}}
  },
  howtostyle/.style={
    colframe=blue!5!black,
    fonttitle=\notoserif\upshape,
    fontupper=\notoserif\itshape,
  },
  rmkstyle/.style={
    colframe=red!10!black
  },
  ELEGANTtitle/.code n args={2}
    {
      \tcbset
        {
          title=
            {\csname #1name\endcsname~%
              \ifdef{\thetcbcounter}{\thetcbcounter}{}%
              \ifblank{#2}{}{\ (#2)}
            }
        }
    },
  % #1 is the command name of the theorem environment
  % #2 is the name of the theorem
  ELEGANTlabel/.code n args={2}
    {
      \ifblank{#2}
        {}{\tcbset{label={#1:#2}}}
    }
}


\NewDocumentCommand \ELEGANTnewtheorem { m m m O{}  }{
  \expandafter\ifblank\expandafter{#4}{
      \tcbset{
        new/usecnt/.style={}
      }
    }{
      \tcbset{
        new/usecnt/.style= {#4}
      }
    }
    \DeclareTColorBox[auto counter,number within=\thmcnt, usecnt]{#1}{ g o t\label g }{ % #1 is the command name of the theorem environment
    parskip, common, #3,
        % #3 is the thmstyle
        IfValueTF={##1}
          {ELEGANTtitle={#1}{##1}}
          {
            IfValueTF={##2}
            {ELEGANTtitle={#1}{##2}}
            {ELEGANTtitle={#1}{}}
          },
          % ##1 is the name of the theorem in tcolorbox format.
          % ##2 is the name of the theorem in amsthm format
        IfValueT={##4}
          { % ##4 is the label in tcolorbox format or the actual label in the command \label{}.
            IfBooleanTF={##3} % ##3 is value if \label{} is used.
              {ELEGANTlabel={##4}}
              {ELEGANTlabel={#2}{##4}}
          }
      }
    \DeclareTColorBox{#1*}{ g o }{
      parskip, common,#3,
        IfValueTF={##1}
          {ELEGANTtitle={#1}{##1}}
          {
            IfValueTF={##2}
            {ELEGANTtitle={#1}{##2}}
            {ELEGANTtitle={#1}{}}
          },
      }
  }

\ELEGANTnewtheorem{definition}{def}{defstyle}

\ELEGANTnewtheorem{theorem}{thm}{thmstyle}[use counter from = definition]%

\ELEGANTnewtheorem{proposition}{prp}{thmstyle}[use counter from = definition]%

\ELEGANTnewtheorem{lemma}{lem}{thmstyle}[use counter from = definition]%

\ELEGANTnewtheorem{corollary}{cor}{thmstyle}[use counter from = definition]%

\ELEGANTnewtheorem{example}{exm}{exmstyle}

\ELEGANTnewtheorem{solution}{}{exmstyle}[no counter]

\ELEGANTnewtheorem{proof}{}{exmstyle}[no counter]

\ELEGANTnewtheorem{howto}{}{howtostyle}[no counter]

\ELEGANTnewtheorem{remark}{}{rmkstyle}[no counter]

\ELEGANTnewtheorem{note}{}{thmstyle}[no counter]

\newcounter{exer}[section]
\setcounter{exer}{0}
\renewcommand{\theexer}{\thesection.\arabic{exer}}

\newenvironment{exercise}[1][]{
  \refstepcounter{exer}
  \par\noindent\makebox[-3pt][r]{
    \footnotesize\color{red!90}\HandPencilLeft\quad}
    \comicneueangular
    \textbf{\color{blue!90}{\exercisename} \theexer ~~ #1}}{
    \par\vspace{\stretch{1}}}
